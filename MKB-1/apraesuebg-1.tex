%%%%%%%%%%%%%%%%%%%%%%%%%%%%%%%%%%%%%%%%%%%%%%%%%%%%%%%%%%%%%%%%%%%%%%%%%%%%%%%
\small
%%%%%%%%%%%%%%%%%%%%%%%%%%%%%%%%%%%%%%%%%%%%%%%%%%%%%%%%%%%%%%%%%%%%%%%%%%%%%%%
\begin{Aufgabe}[Parallelprojektion -- Standardisometrie]
\label{Standardiso}
%%%%%%%%%%%%%%%%%%%%%%%%%%%%%%%%%%%%%%%%%%%%%%%%%%%%%%%%%%%%%%%%%%%%%%%%%%%%%%%
\footnotesize\\[-4ex]
%%%%%%%%%%%%%%%%%%%%%%%%%%%%%%%%%%%%%%%%%%%%%%%%%%%%%%%%%%%%%%%%%%%%%%%%%%%%%%%
Zur Spezifikation einer Parallelprojektion verwenden wir einen Punkt\, $O$\, als Koordinatenursprung und drei \textbf{Einheitspunkte}\, $E_1$,\, $E_2$\, und \,$E_3$\, mit Projektionsbildern\, $O'$,\, $E_1'$,\, $E_2'$\, und\, $E_3'$, \,wie in der Abbildung gezeigt. Das Projektionsbild der\, $x$-Achse (bzw.\, $x_1$-Achse) des Koordinatendreibeins verl"auft in Verl"angerung der Strecke \,$\overline{O'E_1'}$,\, das Projektionsbild der \,$y$-Achse (bzw.\, $x_2$-Achse) in Verl"angerung der Strecke\, $\overline{O'E_2'}$,\, das Bild der \, $z$-Achse (bzw.\, $x_3$-Achse) in Verl"angerung der Strecke\, $\overline{O'E_3'}$.

\textit{Hinweis: Es ist n"utzlich, wenn Sie auch sprachlich zwischen dem \glqq r"aumlichen Koordinatensystem\grqq\ oder \glqq r"aumlichen Dreibein\grqq\ einerseits und dessen Projektionsbild, dem \glqq ebenen Dreibein\grqq\ andererseits unterscheiden.}
%%%%%%%%%%%%%%%%%%%%%%%%%%%%%%%%%%%%%%%%%%%%%%%%%%%%%%%%%%%%%%%%%%%%%%%%%%%%%%%
\def\largeur{14}
\def\hauteur{8}
\def\origine{(7,6)}
%\def\einheit{$\sqrt{3}$}
\begin{center}
%\begin{tikzpicture}[scale = 3]
\begin{tikzpicture}[x=1cm, y=1cm, semitransparent]
%\draw[step=5mm, line width=0.2mm, black!20!white] (0,0) grid (\largeur,\hauteur);
%\draw[step=1cm, line width=0.3mm, black!50!white] (0,0) grid (\largeur,\hauteur);
%\draw[-] (0,0) -- (\frac{\sqrt{3}}{2}, \frac{1}{2});
\draw[-, line width=0.5mm, blue] \origine -- +(90: 1);
\draw[-, line width=0.5mm, red] \origine -- +(210: 1);
\draw[-, line width=0.5mm, green] \origine -- +(330: 1);

\begin{scope}
    \clip(0,0) rectangle (\largeur,\hauteur);

\foreach \i in {-30,...,30}
\draw[-] (7,{6+\i}) --  +(330: 10);
\foreach \i in {-30,...,30}
\draw[-] (7,{6+\i}) --  +(150: 10);

\foreach \i in {-30,...,30}
\draw[-] (7,{6+\i}) --  +(210: 10);

\foreach \i in {-30,...,30}
\draw[-] (7,{6+\i}) --  +(30: 10);

\foreach \i in {-30,...,30}
\draw[-] ({7 + \i * 0.5*sqrt(3) } ,{6-\hauteur}) --  +(0,2*\hauteur);

\end{scope}

\filldraw [black] \origine circle (2.5pt) node[below]{$O'$};
\filldraw [red] {\origine  +(210: 1)} circle (2.5pt) node[below]{$E_1'$};
\filldraw [green] {\origine +(330:1)} circle (2.5pt) node[below]{$E_2'$};
\filldraw [blue] {\origine +(90:1)} circle (2.5pt) node[left]{$E_3'$};
\draw [gray] \origine circle (1);
\end{tikzpicture}
\end{center}
%%%%%%%%%%%%%%%%%%%%%%%%%%%%%%%%%%%%%%%%%%%%%%%%%%%%%%%%%%%%%%%%
\enlargethispage{3ex}
\small
%%%%%%%%%%%%%%%%%%%%%%%%%%%%%%%%%%%%%%%%%%%%%%%%%%%%%%%%%%%%%%%%%%%%%%%%%%%%%%%
\begin{Teilaufgaben}
\setlength{\itemsep}{-0.5ex}
\item \textbf{Zeichnen Sie} in die Skizze die Projektionsbilder der Koordinatenhalbachsen (in Verl"angerung des Koordinatendreibeins) \textbf{ein}.
\item \textbf{Stellen Sie} einen W"urfel mit achsparallelen Kanten der L"ange\, $3$\, (L"angeneinheiten),
%$\SI{3}{\centi\meter}$,
 dessen \glqq linker, hinterer, oberer Eckpunkt\grqq\ bei \,$(8,2,3)$\, liegt, in der hier spezifizierten Parallelprojektion (Standard-Isometrie) \textbf{dar.}
 \\[0.5ex]
{\footnotesize \textit{Hinweise:  Die Angabe \glqq linker, hinterer, oberer Eckpunkt\grqq\ h"angt von der Position des Betrachters ab, Sie d"urfen (und sollen) hier eine sinnvolle Wahl treffen. Stellen Sie sich vor, dass der W"urfel aus Glas ist, so dass auch die \glqq hinteren\grqq\ Kanten sichtbar sind. Stellen Sie die hinteren Kanten mit schw"acherer Strichst"arke oder gestrichelt dar.}}
\item Wie bewerten Sie diese Art der Darstellung bzgl. Anschaulichkeit und Handhabbarkeit?
\item Nehmen Sie an, dass\, $s_3 \,=\, 1$\, gilt.\footnote{In Wahrheit hat\, $s_3$\, einen Wert, der kleiner ist als $1$, wir werden das noch genauer untersuchen. In der vorliegenden Aufgabe bestimmen wir eigentlich nicht\, $s_1$\, und\, $s_2$\, sondern\,  $\tfrac{s_1}{s_3}$ und $\tfrac{s_2}{s_3}$.} Was sind dann die Werte der "ubrigen axonometrischen Angaben\, $s_1$,\, $s_2$,\, $\alpha$\, und \,$\beta$?
\item K"onnen Sie sich vorstellen, wie Projektionsrichtung und Bildebene zum Koordinatensystem liegen m"ussen, damit sich die hier gezeigte Axonometrie ergibt?
\end{Teilaufgaben}
%%%%%%%%%%%%%%%%%%%%%%%%%%%%%%%%%%%%%%%%%%%%%%%%%%%%%%%%%%%%%%%%
\end{Aufgabe}
%SPLITAufgabe
\small
%%%%%%%%%%%%%%%%%%%%%%%%%%%%%%%%%%%%%%%%%%%%%%%%%%%%%%%%%%%%%%%%%%%%%%%%%%%%%%%
\begin{Aufgabe}[Quader in verschiedenen Projektionen\label{Quader_1_verschiedene_PPen}]
Skizzieren Sie \textbf{auf Ihrem eigenen Papier} einen achsenparallelen Quader mit Breite \,$2$,\, H"ohe \,$1$\, und Tiefe \,$3$\, (L"angeneinheiten), dessen linke hintere untere Ecke am Punkt \,$(0,4,0)$\, liegt. Verwenden Sie hierzu
    \begin{itemize}
%    \setlength{\itemsep}{-0.8ex}
    \item eine sogenannte \glqq Kavalierprojektion\grqq\ oder \glqq Kabinettprojektion\grqq\ mit den axonometrischen Angaben \,$\alpha \,=\, \ang{135}$,\;$ \beta \,= \,\ang{90}$,  \;$s_1 \,=\, \tfrac{\sqrt{2}}{2}$, \;$s_2 \,=\, 1$\, und \,$s_3\, =\, 1$,
     \item eine sog. \glqq Milit"arprojektion \grqq\ mit den Angaben \,$\alpha\,= \,\ang{135}$, \;$\beta\,=\, \ang{135}$,\; $s_1 \,= \,1$, $s_2 \,=\, 1$\, und \, $s_3 \,=\, \tfrac{1}{2}$.
\end{itemize}
\end{Aufgabe}
%SPLITLoesung
\begin{Loesung}
Kavalierprojektion:
\begin{figure}[H]
	\centering
	\includegraphics[width=\textwidth]{Grafiken/UE1_P3_1.png}
	\caption{Quader in Kavalierprojektion}
	\label{fig:P3_1}
\end{figure}
Milit"arprojektion:
\begin{figure}[H]
	\centering
	\includegraphics[width=\textwidth]{Grafiken/UE1_P3_2.png}
	\caption{Quader in Milit"arprojektion}
	\label{fig:P3_2}
\end{figure}
\end{Loesung}