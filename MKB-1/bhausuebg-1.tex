%%%%%%%%%%%%%%%%%%%%%%%%%%%%%%%%%%%%%%%%%%%%%%%%%%%%%%%%%%%%%%%%%%%%%%%%%%%%%%%
\small
%%%%%%%%%%%%%%%%%%%%%%%%%%%%%%%%%%%%%%%%%%%%%%%%%%%%%%%%%%%%%%%%%%%%%%%%%%%%%%%
\begin{Aufgabe}[Quader in verschiedenen Projektionen -- Reprise und Fortsetzung von Aufgabe~\ref{Quader_1_verschiedene_PPen}]
Skizzieren Sie einen achsenparallelen Quader mit Breite \,$2$,\, H"ohe \,$1$\, und Tiefe \,$3$\, (L"angeneinheiten), dessen linke hintere untere Ecke am Punkt \,$(0,4,0)$\, liegt. Verwenden Sie hierzu
    \begin{itemize}
%    \setlength{\itemsep}{-0.8ex}
    \item eine sogenannte \glqq Kavalierprojektion\grqq\ oder \glqq Kabinettprojektion\grqq\ mit den axonometrischen Angaben \,$\alpha \,=\, \ang{135}$,\; $\beta \,= \,\ang{90}$,\; $s_1 \,=\, \tfrac{\sqrt{2}}{2}$,\; $s_2 \,=\, 1$\, und\, $s_3\,=\, 1$,
     \item eine sog. \glqq Milit"arprojektion \grqq\ mit den Angaben \,$\alpha\,= \,\ang{135}$, \;$\beta\,=\, \ang{135}$,\; $s_1 \,= \,1$,\, $s_2 \,=\, 1$\, und\, $s_3 \,=\, \tfrac{1}{2}$,
     \item die isometrische Projektion aus Aufgabe~\ref{Standardiso},
     \item die dimetrische Projektion aus Aufgabe~\ref{Dimetrie}.
\end{itemize}
\end{Aufgabe}
%SPLITLoesung
\begin{Loesung}

\begin{figure}[H]
	\centering
	\includegraphics[width=0.6\textwidth]{Grafiken/UE1_H1_A.png}
	\caption{ „Kavalierprojektion“ oder „Kabinettprojektion“}
	\label{fig:H1_A}
\end{figure}

\begin{figure}[H]
	\centering
	\includegraphics[width=0.6\textwidth]{/Grafiken/UE1_H1_B.png}
	\caption{ Milit"arprojektion}
	\label{fig:H1_B}
\end{figure}

\begin{figure}[H]
	\centering
	\includegraphics[width=0.6\textwidth]{Grafiken/UE1_H1_C.png}
	\caption{ Isometrische Projektion}
	\label{fig:H1_C}
\end{figure}

\begin{figure}[H]
	\centering
	\includegraphics[width=0.6\textwidth]{Grafiken/UE1_H1_D.png}
	\caption{ Dimetrische Projektion}
	\label{fig:H1_D}
\end{figure}
\end{Loesung}
%SPLITAufgabe
%%%%%%%%%%%%%%%%%%%%%%%%%%%%%%%%%%%%%%%%%%%%%%%%%%%%%%%%%%%%%%%%%%%%%%%%%%%%%%%
\begin{Aufgabe}[Kuboktaeder] \label{Aufgabe_Kuboktaeder}
\footnotesize\\[-4ex]
Die Einheitsl"ange in dieser Aufgabe sei $\SI{10}{cm}$. Zeichnen Sie einen W"urfel mit Einheitskantenl"ange $\SI{10}{cm}$\, in parallelperspektivischer Darstellung. Um "Uberdeckungen hinten liegender Kanten durch vordere Kanten m"oglichst zu vermeiden, probieren Sie einmal die folgende Angaben aus:\; $\overrightarrow{O'E_1'} = \begin{psmallmatrix} - \SI{6}{cm}\\  - \SI{6}{cm}  \end{psmallmatrix}$,\,
$\overrightarrow{O'E_2'} = \begin{psmallmatrix}  \SI{10}{cm}\\ - \SI{2}{cm}  \end{psmallmatrix}$,\, $\overrightarrow{O'E_3'} = \begin{psmallmatrix}  \SI{0}{cm}\\  \SI{10}{cm}  \end{psmallmatrix}$.
%%%%%%%%%%%%%%%%%%%%%%%%%%%%%%%%%%%%%%%%%%%%%%%%%%%%%%%%%%%%%%%%%%%%%%%%%%%%%%%
\small
\begin{Teilaufgaben}
\item Wie gro"s ist dann der Skalierungsfaktor \, $s_1 = \frac{ \left\| \overrightarrow{O'E_1'} \right\| }{10}$?
\item  \textbf{\glqq Schneiden Sie\grqq} die acht Ecken des W"urfels durch Ebenen \glqq ab\grqq, welche die von den Ecken ausgehenden Kanten halbieren. Wieviele Fl"achen, Ecken und Kanten besitzt der so entstehende sogenannte \textbf{Kuboktaeder}?
\item  "Uberpr"ufen Sie, ob die Anzahl\, $e$\, der Ecken, die Anzahl\, $f$\, der Fl"achen und die Anzahl\, $k$\, der Kanten \textbf{f"ur den Kuboktaeder} die Eulersche Polyederformel erf"ullt: \[e\, -\, k\, +\, f \; =\; 2.\]
\item Bestimmen Sie die Anzahl\, $e_W$\, der Ecken, die Anzahl\, $f_W$\, der Fl"achen und die Anzahl\, $k_W$\, der Kanten  \textbf{f"ur den W"urfel} und "uberpf"ufen Sie auch hier die G"ultigkeit der Polyederformel.
\item Wiederholen Sie die Aufgabe bei Verwendung einer Kavalierprojektion, wobei Sie zum Beispiel\, $\overrightarrow{O'E_1'} = \begin{psmallmatrix} - \SI{2}{cm}\\  - \SI{4}{cm}  \end{psmallmatrix}$\, oder auch\, $\overrightarrow{O'E_1'} = \begin{psmallmatrix} - \SI{4}{cm}\\ - \SI{2}{cm}  \end{psmallmatrix}$\, w"ahlen.
\end{Teilaufgaben}
\end{Aufgabe}
%SPLITLoesung
\begin{Loesung}
\begin{Teilloesungen}

\item Wir berechnen im folgenden den Skalierungsfaktor:
\begin{align*}
s_1 &= \frac{\sqrt{(-2)^{2}+(-4)^{2}}}{10}\\
&= \frac{\sqrt{4+16}}{10}
&= \frac{\sqrt{20}}{10}
&= \frac{2 \cdot \sqrt{5}}{10}
&= \frac{2 \cdot \sqrt{5}}{10}
&\approx 0,45
\end{align*}

\item \textbf{Kuboktaeder:} \\
Ecken($e$): 12\\
Kanten($k$): 24\\
Fl"achen:($f$): 14\\
\abc %c
\begin{align*}
12 - 24 + 14 = 2
\end{align*}

\item \begin{figure}[H]
	\centering
	\includegraphics[width=1\textwidth]{MKB/UE_01/Hausuebungen/Grafiken/UE1_H2_1.png}
	\caption{Ansicht 1}
	\label{fig:H2_1}
\end{figure}

\begin{figure}[H]
	\centering
	\includegraphics[width=1\textwidth]{MKB/UE_01/Hausuebungen/Grafiken/UE1_H2_2.png}
	\caption{Ansicht 2}
	\label{fig:H2_2}
\end{figure}

\end{Teilloesungen}
\end{Loesung}
%SPLITAufgabe
\begin{Aufgabe}[Dimetrische Projektion nach DIN ISO 5456-3 (Ingenieur-Axonometrie)]
\label{Dimetrie}
%%%%%%%%%%%%%%%%%%%%%%%%%%%%%%%%%%%%%%%%%%%%%%%%%%%%%%%%%%%%%%%%%%%%%%%%%%%%%%%
\\[-5ex]
{\def\largeur{14}
\def\hauteur{8}
\def\origine{(7,6)}
\def\alpha{221.41}
\def\beta{352.819}
\def\Sfx{0.5}
\def\Sfy{1}
\def\Sfz{1}
%\def\einheit{$\sqrt{3}$}
\begin{center}
%\begin{tikzpicture}[scale = 3]
\begin{tikzpicture}[x=1cm, y=1cm, semitransparent]
\draw[-, line width=0.5mm, red] \origine -- +(\alpha: \Sfx);
\draw[-, line width=0.5mm, green] \origine -- +(\beta: \Sfy);
\draw[-, line width=0.5mm, blue] \origine -- +(90: \Sfz);

\begin{scope}
    \clip(0,0) rectangle (\largeur,\hauteur);

\foreach \i in {-30,...,30}
\draw[-] ( {7 + \i*\Sfy*cos(\beta)},{6+\i*\Sfy*sin(\beta)}) --  + ( {20* \Sfx* cos(\alpha)}, {20* \Sfx* sin(\alpha)} );
\foreach \i in {-30,...,30}
\draw[-] ( {7 + \i*\Sfy*cos(\beta)},{6+\i*\Sfy*sin(\beta)}) --  + ( {-20* \Sfx* cos(\alpha)}, {-20* \Sfx* sin(\alpha)} );

\foreach \i in {-30,...,30}
\draw[-] ( {7 + \i*\Sfx*cos(\alpha)},{6+\i*\Sfx*sin(\alpha)}) --  + ( {20* \Sfy* cos(\beta)}, {20* \Sfy* sin(\beta)} );
\foreach \i in {-30,...,30}
\draw[-] ( {7 + \i*\Sfx*cos(\alpha)},{6+\i*\Sfx*sin(\alpha)}) --  + ( {-20* \Sfy* cos(\beta)}, {-20* \Sfy* sin(\beta)} );



%\foreach \i in {-30,...,30}
%%\draw[-] { \origine +  ( {\i * \Sfy * cos( {\beta + 180} ) },{\i * \Sfy * sin( {\beta + 180} )})  + (0,-\hauteur)} --  +(0,{2*\hauteur});
%\draw[-] {\origine + \i *({\beta + 180}: \Sfy) + (0,-\hauteur) } --  + (0,{2*\hauteur});

%\foreach \i in {-30,...,30}
%\draw[-] ({7 + \i * \Sfy *cos(\beta ) } ,{6 + \Sfy *sin(\beta )-\hauteur}) --  +(0,2*\hauteur);

\end{scope}

\filldraw [black] \origine circle (2.5pt) node[below]{$O'$};
\filldraw [red] {\origine  +(\alpha: \Sfx)} circle (2.5pt) node[below]{$E_1'$};
\filldraw [green] {\origine +(\beta: \Sfy)} circle (2.5pt) node[below]{$E_2'$};
\filldraw [blue] {\origine +(90: \Sfz)} circle (2.5pt) node[left]{$E_3'$};
\draw [gray] \origine circle (1);

\end{tikzpicture}
\end{center}
}
%%%%%%%%%%%%%%%%%%%%%%%%%%%%%%%%%%%%%%%%%%%%%%%%%%%%%%%%%%%%%%%%
\vspace{1ex}
\small
%%%%%%%%%%%%%%%%%%%%%%%%%%%%%%%%%%%%%%%%%%%%%%%%%%%%%%%%%%%%%%%%%%%%%%%%%%%%%%%
\begin{Teilaufgaben}
%\setlength{\itemsep}{-0.5ex}
\item \textbf{Zeichnen Sie} in die Skizze die Projektionsbilder der Koordinatenhalbachsen \textbf{ein}.
\item \textbf{Stellen Sie} einen Glasw"urfel mit achsparallelen Kanten der L"ange\, $3$\, (L"angeneinheiten),
%$\SI{3}{\centi\meter}$,
dessen linker, hinterer, oberer Eckpunkt bei\, $(2,-4,0)$\, liegt, unter in der hier spezifizierten Parallelprojektion (Dimetrie bzw. Ingenieur-Axonometrie) \textbf{dar}.
\\[0.5ex]
{\footnotesize \textit{Hinweis: Wiederum h"angt die Angabe \glqq linker, hinterer, oberer Eckpunkt\grqq\ vom Betrachter ab, treffen Sie hier eine sinnvolle Wahl!}}
\small
\item Wie bewerten Sie diese Art der Darstellung bzgl. Anschaulichkeit und Handhabbarkeit?
\item Bestimmen Sie die Verh"altnisse\, $\tfrac{s_1}{s_2}$\, und\, $\tfrac{s_3}{s_2}$\, sowie die "ubrigen axonometrische Angaben\, $\alpha$\, und \,$\beta$\, beispielsweise durch Messung mit einem Geodreieck.
\end{Teilaufgaben}
%%%%%%%%%%%%%%%%%%%%%%%%%%%%%%%%%%%%%%%%%%%%%%%%%%%%%%%%%%%%%%%%
\end{Aufgabe}
%SPLITLoesung
\begin{Loesung}
Diese Aufabe wurde in der Vorlesung behandelt.
\end{Loesung}