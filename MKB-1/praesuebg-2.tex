\begin{Aufgabe}[Analyse von Kavalierprojektionen]
\label{Analyse}%
%%%%%%%%%%%%%%%%%%%%%%%%%%%%%%%%%%%%%%%%%%%%%%%%%%%%%%%%%%%%%%%%%%%%%%%%%%%%%%%
\footnotesize%
%%%%%%%%%%%%%%%%%%%%%%%%%%%%%%%%%%%%%%%%%%%%%%%%%%%%%%%%%%%%%%%%%%%%%%%%%%%%%%%
Zur axonometrischen Festlegung einer Kavalierprojektion wird das r"aumliche Dreibein\, $(O, E_1, E_2, E_3)$\, in die $y$-$z$-Koordinatenebene abgebildet (projiziert). Nach Einf"uhrung eines Koordinatensystems\, $\left( O; \vec{u},\vec{v}\right)$\, in dieser Ebene mit\, $\vec{u} = \overrightarrow{O E_2}$\, und\, $\vec{v} = \overrightarrow{O E_3}$\, ergeben sich f"ur die (projizierten) Bildpunkte\, $O'$,\, $E_2'$\, und\, $E_3'$ die  Koordinaten\, $(0,0)$,\, $(1,0)$\, und $(0,1)$, und zwar f"ur jede Projektionsrichtung. Wir "uberlegen uns noch, wo der Bildpunkt\, $E_1'$\, liegt.

Hierzu legen wir eine Projektionsrichtung\, $\left\langle\vec{p}\right\rangle$\, durch Winkel\, $\theta$\, und\, $\varphi$\, fest, wie in Abbildung~\ref{Projektionsrichtung} veranschaulicht. Diese Winkel entsprechen den \glqq geographische Koordinaten\grqq, d.h. dem Längengrad\, $\varphi$\, bzw. dem Breitengrad  \, $\theta$\, des Durchstoßpunktes der Projektionsgeraden durch die im Ursprung zentrierte Einheitskugel. Man kann rechnerisch zeigen, dass der Bildpunkt\, $E_1'$\, die Koordinaten\, $\begin{psmallmatrix} - \tan(\varphi) \\ - \frac{\tan(\theta)}{\cos(\varphi)} \end{psmallmatrix}$\, hat.
%%%%%%%%%%%%%%%%%%%%%%%%%%%%%%%%%%%%%

\begin{figure}[ht]
  \centering
  \includegraphics[width=0.4\textwidth]{../Grafiken/Projektion_Kugelkoordinaten_neu.png}
   %\includegraphics[width=0.3\textwidth]{../Grafiken/ProjektionsRichtung_Kugelkoord.pdf}
  \caption{Zur Festlegung der Projektionsrichtung mit \glqq geografischen\grqq\ Winkeln.}
  \label{Projektionsrichtung}
\end{figure}
%%%%%%%%%%%%%%%%%%%%%%%%%%%%%%%%%%%%%
\begin{Teilaufgaben}
\item Berechnen Sie\, $E_1'$\, und zeichnen Sie das ebene Dreibein f"ur den Fall\, $\theta = \ang{45}$\, und  $\varphi = \ang{45}$.
\item Beschreiben Sie, welche Parallelprojektionen das r"aumliche Dreibein jeweils auf die folgenden ebenen Dreibeine abbilden.
\end{Teilaufgaben}
%%%%%%%%%%%%%%%%%%%%%%%%%%%%%%%%%%%%%
\begin{minipage}{0.3\textwidth}
\includegraphics{../Grafiken/FrontalZweibein.png}
\end{minipage}
\hspace{0.02\textwidth}
\begin{minipage}{0.3\textwidth}
\includegraphics{../Grafiken/FrontalDreibein.png}
\end{minipage}
\hspace{0.02\textwidth}
\begin{minipage}{0.3\textwidth}
\includegraphics[width=\textwidth]{../Grafiken/KavalierDreibein_mit_Gitter.png}
\end{minipage}

%%%%%%%%%%%%%%%%%%%%%%%%%%%%%%%%%%%%%%%%%%%%%%%%%%%%%%%%%%%%%%%%%%%%%%%%%%%%%%%
\footnotesize%
%%%%%%%%%%%%%%%%%%%%%%%%%%%%%%%%%%%%%%%%%%%%%%%%%%%%%%%%%%%%%%%%%%%%%%%%%%%%%%%
\textit{Hinweis: Zur Analyse der Grafik ganz rechts k"onnen Sie auf die Gleichung\,  $\begin{psmallmatrix} - \tan(\varphi) \\ - \frac{\tan(\theta)}{\cos(\varphi)} \end{psmallmatrix}\, =\, \begin{psmallmatrix} -\tfrac{1}{2}\\ -\tfrac{1}{2}\end{psmallmatrix}$\, zur"uckgreifen, vgl. Aufgabe~\ref{Analyse}. Alternativ hierzu k"onnten Sie die Projektionsrichtung in dem vorliegenden (speziellen) Fall so angeben wie in Aufgabe~\ref{Analyse_KavProj}. Dort werden allerdings \textbf{nicht} die Winkel\, $\theta$\, und\, $\varphi$\, verwendet.}


\textbf{L"osung:}\\
Aus der angegebenen Gleichung folgt\, $\tan(\varphi)\, =\, \tfrac{1}{2}$\, somit\, $\cos(\varphi)\, =\, \tfrac{2}{\sqrt{5}}$\, und\, $\sin(\varphi)\, =\, \tfrac{1}{\sqrt{5}}$. Damit folgt weiter \, $\tan(\theta)\, =\, \tfrac{1}{2}\cdot \tfrac{2}{\sqrt{5}}\, =\, \tfrac{1}{\sqrt{5}}$, \, somit\, $\cos(\theta)\, =\, \tfrac{\sqrt{5}}{\sqrt{6}}$\, und $\sin(\theta)\, =\, \tfrac{1}{\sqrt{6}}$. Mit der Formel f"ur Kugelkoordinaten ergibt sich somit der Projektionsvektor \[\vec{p}\; =\; \begin{pmatrix} \cos(\theta)\, \cos(\varphi)\\  \cos(\theta)\, \sin(\varphi)\\ \sin(\theta)\end{pmatrix}\;= \; \begin{pmatrix} \tfrac{\sqrt{2}}{\sqrt{3}}\\  \tfrac{1}{2} \cdot \tfrac{\sqrt{2}}{\sqrt{3}}\\ \tfrac{1}{2} \cdot \tfrac{\sqrt{2}}{\sqrt{3}}\end{pmatrix}. \]

In Haus"ubung~\ref{Analyse_KavProj} wird ein Winkel\, $\delta$\, verwendet, der die Neigung des Projektionsrichtung zur $x_2$-$x_3$-Ebene angibt.
Hier ergibt sich die Forderung\, $\tan(\delta)\, =\, \frac{1}{\tfrac{1}{\sqrt{2}}}\, =\, \sqrt{2}$. Hieraus ergibt sich\, $\cos(\delta)\, =\, \tfrac{\sqrt{2}}{\sqrt{3}}$\, und\, $\sin(\delta)\,= \, \tfrac{1}{\sqrt{3}}$. Somit ergibt sich f"ur den Projektionsvektor
\[\vec{p}\; =\; \begin{pmatrix} \sin(\delta)\\  \cos(\delta)\, \cos\left(\ang{45}\right)\\ \cos(\delta)\, \sin\left(\ang{45}\right)\end{pmatrix}\;= \; \begin{pmatrix} \tfrac{\sqrt{2}}{\sqrt{3}}\\  \tfrac{1}{2} \cdot \tfrac{\sqrt{2}}{\sqrt{3}}\\ \tfrac{1}{2} \cdot \tfrac{\sqrt{2}}{\sqrt{3}}\end{pmatrix}. \]
%%%%%%%%%%%%%%%%%%%%%%%%%%%%%%%%%%%%%%
\end{Aufgabe}
%SPLITLoesung
\begin{Loesung}
\begin{Teilloesungen}
\usetikzlibrary{calc}

\item Wir berechnen $E_1'$ mit der gegebenen 	Formel: \\
	\begin{align*}
		E_1' &= \begin{pmatrix}- \tan(45)\\  \frac{- \tan(45)}{{\cos(45)}}\end{pmatrix}\\	
		&= \begin{pmatrix}-1\\  \frac{-1}{\frac{\sqrt{2}}{2}}\end{pmatrix}\\
		&= \begin{pmatrix}-1\\  \frac{-1}{\frac{1}{\sqrt{2}}}\end{pmatrix}\\	
		&= \begin{pmatrix}-1\\ \sqrt{2} \end{pmatrix}
	\end{align*}

\begin{figure}[H]
	\centering
	\includegraphics[width=0.8\textwidth]{MKB/UE_02/Praesenzuebungen/Grafiken/UE2_P4.png}
	\caption{Projizierter Punkt$E_1'$.}
	\label{fig.P4}
\end{figure}
	
\item Keine L"osung darstellbar, hier geht es um euer Verst"andnis für die verschiedenen Perspektiven
\end{Teilloesungen}
\end{Loesung}
%SPLITAufgabe
\begin{Aufgabe}[Analyse von Kavalierprojektionen -- Fortsetzung]
\textbf{F"ur Schnelle zugleich, f"ur alle anderen zuhause:}
\begin{Teilaufgaben}
\item F"ur welche Kombination von\, $\theta$\, und  $\varphi$ ergibt sich die folgende Axonometrie:
 \begin{figure}[ht]
  \centering
  \includegraphics[width=0.3\textwidth]{../Grafiken/Kabinettprojektion-3a.png}
  \caption{Kabinett- bzw. Kavalierprojektion mit\, $\alpha = \ang{135}$\, und \,$s_1\, =\,1$.}
  \label{Kabinettproj}
  \end{figure}

  Hinweis: "Uberzeugen Sie sich zun"achst davon, dass der rote Vektor in Abbildung~\ref{Kabinettproj} die Koordinaten\, $\begin{psmallmatrix} - \frac{1}{\sqrt{2}}\\ - \frac{1}{\sqrt{2}} \end{psmallmatrix}$\, besitzt. Bestimmen Sie dann Werte f"ur\, $\theta$\, und\, $\varphi$\, so, dass die Gleichung \[\begin{pmatrix} - \tan(\varphi) \\ - \frac{\tan(\theta)}{\cos(\varphi)} \end{pmatrix}\;=\;\begin{pmatrix} - \frac{1}{\sqrt{2}}\\ - \frac{1}{\sqrt{2}} \end{pmatrix}\] erf"ullt ist.
\end{Teilaufgaben}
\end{Aufgabe}
%SPLITLoesung
\begin{Loesung}
\usetikzlibrary{calc}

	F"ur die gegebene Axonometrie ist $\overrightarrow{O'E_1'}\,=\,\begin{pmatrix}- \frac{1}{\sqrt{2}}\\- \frac{1}{\sqrt{2}}\end{pmatrix}$. Wir bestimmen Werte f"ur $\theta$ und $\varphi$ so, dass
	\[\begin{pmatrix}- \tan (\varphi)\\
	- \frac{\tan (\theta)}{\cos (\varphi)}\end{pmatrix}\;=\;\begin{pmatrix}- \frac{1}{\sqrt{2}}\\- \frac{1}{\sqrt{2}}\end{pmatrix}. 
	\] Zu l"osen sind also die Gleichungen \begin{align*}- \tan (\varphi)\;&=\;- \frac{1}{\sqrt{2}}\\ 
	- \frac{\tan (\theta)}{\cos (\varphi)}\;&=\;- \frac{1}{\sqrt{2}}
	\end{align*} Aus der ersten Gleichung folgt $\varphi\,=\,\arctan\left(\frac{1}{\sqrt{2}}\right)\,=\,\ang{35.26}$. Wir setzen diesen Wert in die zweite Gleichung ein: 
	\[- \frac{\tan (\theta)}{\cos \left(\ang{35.26}\right)}\;=\;- \frac{1}{\sqrt{2}}\] Wir l"osen in zwei Schritten nach $\theta$ auf: 
	\[\tan (\theta)\;=\;\frac{\cos \left(\ang{35.26}\right)}{\sqrt{2}}\] Hieraus folgt schlie"slich:
	\[\theta\;=\;\arctan\left(\frac{\cos \left(\ang{35.26}\right)}{\sqrt{2}}\right)\;=\;\ang{30}
	\]
\end{Loesung}