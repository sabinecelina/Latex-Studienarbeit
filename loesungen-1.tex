%ID: 1 -- Loesungen zu: Parallelprojektion -- Standardisometrie
NULL
%ID: 3 -- Loesungen zu: Quader in verschiedenen Projektionen -- Reprise und Fortsetzung von Aufgabe~\ref{Quader_1_verschiedene_PPen}

\begin{Loesung}

\begin{figure}[H]
	\centering
	\includegraphics[width=0.6\textwidth]{Grafiken/UE1_H1_A.png}
	\caption{ „Kavalierprojektion“ oder „Kabinettprojektion“}
	\label{fig:H1_A}
\end{figure}

\begin{figure}[H]
	\centering
	\includegraphics[width=0.6\textwidth]{/Grafiken/UE1_H1_B.png}
	\caption{ Milit"arprojektion}
	\label{fig:H1_B}
\end{figure}

\begin{figure}[H]
	\centering
	\includegraphics[width=0.6\textwidth]{Grafiken/UE1_H1_C.png}
	\caption{ Isometrische Projektion}
	\label{fig:H1_C}
\end{figure}

\begin{figure}[H]
	\centering
	\includegraphics[width=0.6\textwidth]{Grafiken/UE1_H1_D.png}
	\caption{ Dimetrische Projektion}
	\label{fig:H1_D}
\end{figure}
\end{Loesung}
%
%ID: 12 -- Loesungen zu: G'unstige und weniger g'unstige Parallelperspektiven

\begin{Loesung}
\usetikzlibrary{calc}
\begin{Teilloesungen}
\item Militärprojektion zeichnen:
	\begin{figure}[H]
		\centering
		\includegraphics[width=0.5\textwidth]{Grafiken/UE2_T1_a.png}
		\caption{Milit"arprojektion des Quaders}
	\label{fig:T1_1}
	\end{figure}
	

\item	Parallelprojektion zeichnen:
	\begin{figure}[H]
		\centering
		\includegraphics[width=0.5\textwidth]{Grafiken/UE2_T1_b.png}
		\caption{Ung"unstige Projektion des Quaders, da die x- und y- Achse aufeinander liegen und der Quader so nicht mehr erkennbar ist. Die Vorderseite des Quaders wurde in Gelb markiert, die R"uckseite in Gr"un.}
	\label{fig:T1_2}
	\end{figure}
\end{Loesung}
%
