%ID: 2 -- Loesungen zu: %Quader in verschiedenen Projektionen\label{Quader_1_verschiedene_PPen}
\begin{Loesung}
Kavalierprojektion:
\begin{figure}[H]
	\centering
	\includegraphics[width=\textwidth]{Grafiken/UE1_P3_1.png}
	\caption{Quader in Kavalierprojektion}
	\label{fig:P3_1}
\end{figure}
Milit"arprojektion:
\begin{figure}[H]
	\centering
	\includegraphics[width=\textwidth]{Grafiken/UE1_P3_2.png}
	\caption{Quader in Milit"arprojektion}
	\label{fig:P3_2}
\end{figure}
\end{Loesung}
%ID: 4 -- Loesungen zu: %Kuboktaeder
\begin{Loesung}
\begin{Teilloesungen}

\item Wir berechnen im folgenden den Skalierungsfaktor:
\begin{align*}
s_1 &= \frac{\sqrt{(-2)^{2}+(-4)^{2}}}{10}\\
&= \frac{\sqrt{4+16}}{10}
&= \frac{\sqrt{20}}{10}
&= \frac{2 \cdot \sqrt{5}}{10}
&= \frac{2 \cdot \sqrt{5}}{10}
&\approx 0,45
\end{align*}

\item \textbf{Kuboktaeder:} \\
Ecken($e$): 12\\
Kanten($k$): 24\\
Fl"achen:($f$): 14\\
\abc %c
\begin{align*}
12 - 24 + 14 = 2
\end{align*}

\item \begin{figure}[H]
	\centering
	\includegraphics[width=1\textwidth]{MKB/UE_01/Hausuebungen/Grafiken/UE1_H2_1.png}
	\caption{Ansicht 1}
	\label{fig:H2_1}
\end{figure}

\begin{figure}[H]
	\centering
	\includegraphics[width=1\textwidth]{MKB/UE_01/Hausuebungen/Grafiken/UE1_H2_2.png}
	\caption{Ansicht 2}
	\label{fig:H2_2}
\end{figure}

\end{Teilloesungen}
\end{Loesung}
%
%ID: 13 -- Loesungen zu: %Pyramidenstumpf in Kavalierprojektion
\begin{Loesung}
\usetikzlibrary{calc}
	Pyramidenstumpf zeichnen:
	\begin{figure}[H]
		\centering
		\includegraphics[width=0.5\textwidth]{Grafiken/UE2_T2_1.png}
		\caption{Weg 1}
	\label{fig:T2_1}
	\end{figure}
	\begin{figure}[H]
		\centering
		\includegraphics[width=0.5\textwidth]{Grafiken/UE2_T2_2.png}
		\caption{Weg 2}
	\label{fig:T2_2}
	\end{figure}
\end{Loesung}
%
