
\small
%%%%%%%%%%%%%%%%%%%%%%%%%%%%%%%%%%%%%%%%%%%%%%%%%%%%%%%%%%%%%%%%%%%%%%%%%%%%%%%
\begin{Aufgabe}[G"unstige und weniger g"unstige Parallelperspektiven]
\begin{Teilaufgaben}
\item Skizzieren Sie einen achsenparallelen Quader mit Breite \,$2$,\, H"ohe \,$4$\, und Tiefe \,$3$\, (L"angeneinheiten), dessen linke hintere untere Ecke am Punkt\, $(0,4,0)$\, liegt. Verwenden Sie hierzu eine sog. \glqq Milit"arprojektion\grqq\ mit den Angaben \,$\alpha\,= \,\ang{135}$, \;$\beta\,=\, \ang{135}$,\; $s_1 \,= \,1$, $s_2 \,=\, 1$\, und \, $s_3 \,=\, \tfrac{1}{2}$.
\item Skizzieren Sie den Quader nun in einer Parallelprojektion mit den axonometrischen
Angaben\, $\alpha= \ang{90}$,\, $\beta= \ang{90}$,\, $s_1 = 1$,\, $s_2 = 1$,\, $s_3 = 1$. Ist dies in punkto Anschaulichkeit eine g"unstige Projektion?
\end{Teilaufgaben}
\end{Aufgabe}
%
\begin{Loesung}
\usetikzlibrary{calc}
\begin{Teilloesungen}
\item Milit\arprojektion zeichnen:
	\begin{figure}[H]
		\centering
		\includegraphics[width=0.5\textwidth]{Grafiken/UE2_T1_a.png}
		\caption{Milit"arprojektion des Quaders}
	\label{fig:T1_1}
	\end{figure}
	

\item	Parallelprojektion zeichnen:
	\begin{figure}[H]
		\centering
		\includegraphics[width=0.5\textwidth]{Grafiken/UE2_T1_b.png}
		\caption{Ung"unstige Projektion des Quaders, da die x- und y- Achse aufeinander liegen und der Quader so nicht mehr erkennbar ist. Die Vorderseite des Quaders wurde in Gelb markiert, die R"uckseite in Gr"un.}
	\label{fig:T1_2}
	\end{figure}
\end{Loesung}
%
%%%%%%%%%%%%%%%%%%%%%%%%%%%%%%%%%%%%%%%%%%%%%%%%%%%%%%%%%%%%%%%%%%%%%%%%%%%%%%%
\begin{Aufgabe}[Pyramidenstumpf in Kavalierprojektion]
%%%%%%%%%%%%%%%%%%%%%%%%%%%%%%%%%%%%%%%%%%%%%%
Zeichnen Sie einen Pyramidenstumpf mit quadratischem Grundriss (die Kantenl"ange d"urfen Sie w"ahlen, 4 Einheiten k"onnte eine gute Wahl sein). Die Spitze der vollst"andigen Pyramide l"age im Punkte\, $S=(0,0,5)$.\, Der Boden des Pyramidenstumpfs liegt in der $x_1$-$x_2$-Ebene, der Deckel $3$ Einheiten dar"uber. Verwenden Sie hierzu  die Kavalierprojektion mit den axonometrischen Angaben \,$\alpha \,=\, \ang{135}$,\;$ \beta \,= \,\ang{90}$,  \;$s_1 \,=\, \tfrac{\sqrt{2}}{2}$, \;$s_2 \,=\, 1$\, und \,$s_3\, =\, 1$
\end{Aufgabe}
%
\begin{Loesung}
\usetikzlibrary{calc}
	Pyramidenstumpf zeichnen:
	\begin{figure}[H]
		\centering
		\includegraphics[width=0.5\textwidth]{Grafiken/UE2_T2_1.png}
		\caption{Weg 1}
	\label{fig:T2_1}
	\end{figure}
	\begin{figure}[H]
		\centering
		\includegraphics[width=0.5\textwidth]{Grafiken/UE2_T2_2.png}
		\caption{Weg 2}
	\label{fig:T2_2}
	\end{figure}
\end{Loesung}
%
%\newpage
%%%%%%%%%%%%%%%%%%%%%%%%%%%%%%%%%%%%%%%%%%%%%%%%%%%%%%%%%%%%%%%%%%%%%%%%%%%%%%%
\begin{Aufgabe}[Parallelprojektionen in die Aufrissebene]
\footnotesize
Wir wollen noch einmal ebene Dreibeine\, $(O', E_1', E_2', E_3')$\, untersuchen, die durch Projektion eines kartesischen r"aumlichen Dreibeins\, $(O, E_1, E_2, E_3)$\, auf die $x_2$-$x_3$-Ebene entstehen. Im projizierten Bild zeigt die $x_2'$-Achse jeweils nach rechts, die $x_3'$-Achse nach oben. Um die
Projektionsrichtung im Raum anzugeben, verwenden wir \glqq geographische Koordinaten\grqq, d.h. den L\angengrad \,$\varphi$\, und den Breitengrad  \,$\theta$\, des Durchstoßpunktes der Projektionsgeraden durch die im Ursprung zentrierte Einheitskugel (vgl. die Abbildung~\ref{Projektionsrichtung}).
\small
%%%%%%%%%%%%%%%%%%%%%%%%%%%%%%%%%%%%%%%

Beschreiben Sie, welche Parallelprojektionen das r"aumliche Dreibein jeweils auf die folgenden ebenen Dreibeine abbilden.\\
\begin{minipage}{0.33\textwidth}
\includegraphics{../Grafiken/FrontalZweibein.png}
\end{minipage}
\hspace{0.02\textwidth}
\begin{minipage}{0.3\textwidth}
\includegraphics{../Grafiken/FrontalDreibein.png}
\end{minipage}
\hspace{0.02\textwidth}
\begin{minipage}{0.3\textwidth}
\includegraphics[width=\textwidth]{../Grafiken/KavalierDreibein_mit_Gitter.png}
\end{minipage}

\textit{Hinweis: Zur Analyse der Grafik ganz rechts k"onnen Sie auf die Gleichung\,  $\begin{psmallmatrix} - \tan(\varphi) \\ - \frac{\tan(\theta)}{\cos(\varphi)} \end{psmallmatrix}\, =\, \begin{psmallmatrix} -\tfrac{1}{2}\\ -\tfrac{1}{2}\end{psmallmatrix}$\, zur"uckgreifen, vgl. Aufgabe~\ref{Analyse}. Alternativ hierzu k"onnen Sie die Projektionsrichtung so angeben wie in Haus"ubung~\ref{Analyse_KavProj}. Dort werden allerdings \textbf{nicht} die Winkel\, $\theta$\, und\, $\varphi$\, verwendet.}
%%%%%%%%%%%%%%%%%%%%%%%%%%%%%%%%%%%%%%%%%%%%%%%%%%%%%%%%%%%%%%%%%%%%%%%%%%%%%%%
\begin{Teilaufgaben}
\item F"ur welche Kombination von\, $\theta$\, und  $\varphi$ ergibt sich die folgende Axonometrie:
 \begin{figure}[ht]
  \centering
  \includegraphics[width=0.3\textwidth]{../Grafiken/Kabinettprojektion-3a.png}
  \caption{Kabinett- bzw. Kavalierprojektion mit\, $\alpha = \ang{135}$\, und \,$s_1\, =\,1$.}
  \label{KabinettprojTut}
  \end{figure}

  Hinweis: "Uberzeugen Sie sich zun"achst davon, dass der rote Vektor in Abbildung~\ref{KabinettprojTut} die Koordinaten\, $\begin{psmallmatrix} - \frac{1}{\sqrt{2}}\\ - \frac{1}{\sqrt{2}} \end{psmallmatrix}$\, besitzt. Bestimmen Sie dann Werte f"ur\, $\theta$\, und\, $\varphi$\, so, dass die Gleichung \[\begin{pmatrix} - \tan(\varphi) \\ - \frac{\tan(\theta)}{\cos(\varphi)} \end{pmatrix}\;=\;\begin{pmatrix} - \frac{1}{\sqrt{2}}\\ - \frac{1}{\sqrt{2}} \end{pmatrix}\] erf"ullt ist.
\end{Teilaufgaben}
%%%%%%%%%%%%%%%%%%%%%%%%%%%%%%%%%%%%%%%%%%%%%%%%%%%%%%%%%%%%%%%%%%%%%%%%%%%%%%%
\end{Aufgabe}
%
\begin{Loesung}
\begin{figure}[H]
	\centering
<	\includegraphics[width=.5\textwidth]{Grafiken/UE2_T3_1.png}
	\caption{\; \ensuremath{\theta =} 0$^{\circ}$ \; \ensuremath{\varphi = 0}$^{\circ}$ \\
}
	\label{T3_1}
\end{figure}

\begin{figure}[H]
	\centering
	\includegraphics[width=.5\textwidth]{Grafiken/UE2_T3_2.png}
	\caption{\; \ensuremath{\theta =} 0$^{\circ}$ \; \ensuremath{\varphi = 45}$^{\circ}$ \\
}
	\label{T3_2}
\end{figure}

\begin{figure}[H]
	\centering
	\includegraphics[width=.5\textwidth]{Grafiken/UE2_T3_3.png}
	\caption{\; \ensuremath{\theta =} 24.09$^{\circ}$ \; \ensuremath{\varphi = 26.56}$^{\circ}$ \\
}
	\label{T3_3}
\end{figure}
\end{Loesung}