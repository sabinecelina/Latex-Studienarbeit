%%%%%%%%%%%%%%%%%%%%%%%%%%%%%%%%%%%%%%%%%%%%%%%%%%%%%%%%%%%%%%%%%%%%%%%%%%%%%%%
\begin{Aufgabe}[Kuboktaeder] \label{Aufgabe_Kuboktaeder}
\footnotesize\\[-4ex]
Die Einheitsl"ange in dieser Aufgabe sei $\SI{10}{cm}$. Zeichnen Sie einen W"urfel mit Einheitskantenl"ange $\SI{10}{cm}$\, in parallelperspektivischer Darstellung. Um "Uberdeckungen hinten liegender Kanten durch vordere Kanten m"oglichst zu vermeiden, probieren Sie einmal die folgende Angaben aus:\; $\overrightarrow{O'E_1'} = \begin{psmallmatrix} - \SI{6}{cm}\\  - \SI{6}{cm}  \end{psmallmatrix}$,\,
$\overrightarrow{O'E_2'} = \begin{psmallmatrix}  \SI{10}{cm}\\ - \SI{2}{cm}  \end{psmallmatrix}$,\, $\overrightarrow{O'E_3'} = \begin{psmallmatrix}  \SI{0}{cm}\\  \SI{10}{cm}  \end{psmallmatrix}$.
%%%%%%%%%%%%%%%%%%%%%%%%%%%%%%%%%%%%%%%%%%%%%%%%%%%%%%%%%%%%%%%%%%%%%%%%%%%%%%%
\small
\begin{Teilaufgaben}
\item Wie gro"s ist dann der Skalierungsfaktor \, $s_1 = \frac{ \left\| \overrightarrow{O'E_1'} \right\| }{10}$?
\item  \textbf{\glqq Schneiden Sie\grqq} die acht Ecken des W"urfels durch Ebenen \glqq ab\grqq, welche die von den Ecken ausgehenden Kanten halbieren. Wieviele Fl"achen, Ecken und Kanten besitzt der so entstehende sogenannte \textbf{Kuboktaeder}?
\item  "Uberpr"ufen Sie, ob die Anzahl\, $e$\, der Ecken, die Anzahl\, $f$\, der Fl"achen und die Anzahl\, $k$\, der Kanten \textbf{f"ur den Kuboktaeder} die Eulersche Polyederformel erf"ullt: \[e\, -\, k\, +\, f \; =\; 2.\]
\item Bestimmen Sie die Anzahl\, $e_W$\, der Ecken, die Anzahl\, $f_W$\, der Fl"achen und die Anzahl\, $k_W$\, der Kanten  \textbf{f"ur den W"urfel} und "uberpf"ufen Sie auch hier die G"ultigkeit der Polyederformel.
\item Wiederholen Sie die Aufgabe bei Verwendung einer Kavalierprojektion, wobei Sie zum Beispiel\, $\overrightarrow{O'E_1'} = \begin{psmallmatrix} - \SI{2}{cm}\\  - \SI{4}{cm}  \end{psmallmatrix}$\, oder auch\, $\overrightarrow{O'E_1'} = \begin{psmallmatrix} - \SI{4}{cm}\\ - \SI{2}{cm}  \end{psmallmatrix}$\, w"ahlen.
\end{Teilaufgaben}
\end{Aufgabe}
%