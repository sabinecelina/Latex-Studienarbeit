\small
%%%%%%%%%%%%%%%%%%%%%%%%%%%%%%%%%%%%%%%%%%%%%%%%%%%%%%%%%%%%%%%%%%%%%%%%%%%%%%%
\begin{Aufgabe}[Castellum\label{Castellum}]
Im Folgenden sind der Grundriss sowie der Aufriss eines Castells gegeben. Fertigen Sie hieraus eine parallelperspektivische Darstellung in Milit"arperspektive an.

\includegraphics[width=0.55\textwidth]{../Grafiken/Castellum_Grundriss}

\includegraphics[width=0.55\textwidth]{../Grafiken/Castellum_Aufriss}
\end{Aufgabe}
%\newpage
\begin{Aufgabe}[Dynamische ver"anderbare Darstellungen mit Geogebra]
\begin{Teilaufgaben}
\item Erzeugen Sie eine Geogebra-Datei, in der Sie die axonometrischen Angaben flexibel einstellen k"onnen.
\item Verwenden Sie diese Geogebra-Datei als Grundlage f"ur Darstellungen des \glqq Castellum\grqq von Aufgabe~\ref{Castellum}. Da die axonometrischen Angaben mit Hilfe der \glqq Schieberegler\grqq\ interaktiv ver"anderbar sind, k"onnen Sie viele verschiedene parallelperspektivische Darstellungen ausprobieren.
\end{Teilaufgaben}
\end{Aufgabe}
%
%%%%%%%%%%%%%%%%%%%%%%%%%%%%%%%%%%%%%%%%%%%%%%%%%%%%%%%%%%%%%%%%%%%%%%%%%%%%%%%
%\newpage

%%%%%%%%%%%%%%%%%%%%%%%%%%%%%%%%%%%%%%%%%%%%%%%%%%%%%%%%%%%
\begin{Aufgabe}[Koordinatenquader -- Koordinatenbestimmung]
Übertragen Sie die folgende Figur auf Ihr Papier, zeichnen Sie den Koordinatenquader von P und lesen
Sie die Koordinaten des Punktes P ab. Der eingezeichnete Punkt P’ soll in der Grundrissebene liegen.\\
\includegraphics[width=0.3\textwidth]{../Grafiken/Koordinatenquader-1.pdf}
%%%%%%%%%%%%%%%%%%%%%%%%%%%%%%%%%%%%%%%%%%%%%%%%%%%%%%%%%%%

Übertragen Sie die folgende Figur auf Ihr Papier, zeichnen Sie den Koordinatenquader von P und lesen
Sie die Koordinaten des Punktes P ab. Der eingezeichnete Punkt P’ soll in der Grundrissebene liegen.\\
\includegraphics[width=0.3\textwidth]{../Grafiken/Koordinatenquader-2.pdf}
\end{Aufgabe}
%
%%%%%%%%%%%%%%%%%%%%%%%%%%%%%%%%%%%%%%%%%%%%%%%%%%%%%%%%%%%%%%%%%%%%%%%%%%%%%%%
\begin{Aufgabe}[Analyse einiger Kavalierprojektionen]
\label{Analyse_KavProj}%
%%%%%%%%%%%%%%%%%%%%%%%%%%%%%%%%%%%%%%%%%%%%%%%%%%%%%%%%%%%%%%%%%%%%%%%%%%%%%%%
\textbf{Untersuchen Sie} einige spezielle Kavalierprojektionen. Die Bildebene ist jeweils die $x_2$-$x_3$-Ebene. Die Projektionsrichtung liegt immer in derjenigen Ebene , welche die $x_1$-Achse enth"alt und den Winkel zwischen der $x_2$-Achse und der $x_3$-Achse halbiert. Diese Ebene ist durch die Gleichung \,$x_2\,=\,x_3$ gegeben. Im Folgenden bezeichne $E_1'$ stets den Schnittpunkt der durch \,$E_1$\, verlaufenden Projektionsgeraden mit der Bildebene. Bei den hier untersuchten Parallelprojektionen ergibt sich stets \,$\alpha = \ang{45}$, der Skalierungsfaktor \,$s_1$\, variiert je nachdem, wie flach oder steil die Projektionsrichtung gew"ahlt wird.
\begin{Teilaufgaben}
\item Die Projektionsrichtung sei um $\ang{45}$ zur \,$x_2$-$x_3$-Ebene geneigt. Bestimmen Sie die L"ange der Strecke \,$s_1= \overline{OE_1'}$.
\item Die Projektionsgerade habe die Steigung \,$m_2 = \frac{2}{1}$.\, Bestimmen Sie die L"ange der Strecke \,$s_1=\overline{OE_1'}$\, sowie den Neigungswinkel \,$\delta_2$\, der Projektionsgeraden zur \,$x_2$-$x_3$-Ebene.
\item Die Projektionsgerade habe die Steigung \,$m_3 = \frac{1}{2}$.\, Bestimmen Sie die L"ange der Strecke \,$s_1=\overline{OE_1'}$\, sowie den Neigungswinkel \,$\delta_3$\, der Projektionsgeraden zur \,$x_2$-$x_3$-Ebene.
\item In der Schule haben Sie zur Darstellung  r"aumlicher Objekte m"oglicherweise immer wieder eine Axonometrie verwendet, bei der \,$s_1\,=\,\tfrac{\sqrt{2}}{2}$\, und \,$\alpha = \ang{45}$\, galt. Welchen Neigungswinkel $\delta$ muss die Projektionsgerade zur $x_2$-$x_3$-Ebene haben, damit sich \, $s_1\,=\,\tfrac{\sqrt{2}}{2}$\, ergibt?
\end{Teilaufgaben}
%%%%%%%%%%%%%%%%%%%%%%%%%%%%%%%%%%%%%%%%%%%%%%%%%%%%%%%%%%%%%%%%
\end{Aufgabe}
%
%%%%%%%%%%%%%%%%%%%%%%%%%%%%%%%%%%%%%%%%%%%%%%%%%%%%%%%%%%%%%%%%
\begin{Aufgabe}[Fortsetzung der Analyse von Aufgabe~\ref{Analyse_KavProj} -- Projektion auf verschiedene Ebenen\label{KavProj_AndereBildebene}]
\begin{Teilaufgaben}
\item Die Bildebene werden nun so \glqq tiefergelegt\grqq, dass sie \textbf{parallel zur}\, $x_2$-$x_3$-Ebene ist und den Punkt\, $(-1,0,0)$\, enth"alt. Untersuchen Sie, wie die Punkte\, $O$,\, und\, $E_1$\, projiziert werden, wenn die Projektionsrichtung so ist wie in der ersten Teilaufgabe von \ref{Analyse_KavProj}. Ist der Skalierungsfaktor\, $s_1$\, im Vergleich zur Aufgabe~\ref{Analyse_KavProj} ver"andert?
\item Was ver"andert sich, wenn Sie die Bildebene\, $\pi$\, um eine weitere L"angeneinheit \glqq tieferlegen\grqq, so dass sie den Punkt\, $(-2,0,0)$\, enth"alt.
\item Versuchen Sie einen Ergebnissatz zu formulieren, der die von Ihnen beobachteten Sachverhalte bei Parallelprojektionen zusammenfasst:\\[1ex]
\textit{Wenn man bei gegebenem r"aumlichen Dreibein \, $\bigl(\,O;\; E_1,\, E_2,\, E_3\,\bigr)$\, und gegebener Projektionsrichtung unterschiedliche zueinander parallel liegende Bildebenen w"ahlt, so ver"andert sich zwar die Lage \ldots, die \ldots bleiben jedoch unver"andert.}
\end{Teilaufgaben}
%%%%%%%%%%%%%%%%%%%%%%%%%%%%%%%%%%%%%%%%%%%%%%%%%%%%%%%%%%%%%%%%
\end{Aufgabe}
%