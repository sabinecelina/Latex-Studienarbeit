
\small
%%%%%%%%%%%%%%%%%%%%%%%%%%%%%%%%%%%%%%%%%%%%%%%%%%%%%%%%%%%%%%%%%%%%%%%%%%%%%%%
\begin{Aufgabe}[Castellum\label{Castellum}]
Im Folgenden sind der Grundriss sowie der Aufriss eines Castells gegeben. Fertigen Sie hieraus eine parallelperspektivische Darstellung in Milit"arperspektive an.

\includegraphics[width=0.55\textwidth]{../Grafiken/Castellum_Grundriss}

\includegraphics[width=0.55\textwidth]{../Grafiken/Castellum_Aufriss}
\end{Aufgabe}
%\newpage
\begin{Aufgabe}[Dynamische ver"anderbare Darstellungen mit Geogebra]
\begin{Teilaufgaben}
\item Erzeugen Sie eine Geogebra-Datei, in der Sie die axonometrischen Angaben flexibel einstellen k"onnen.
\item Verwenden Sie diese Geogebra-Datei als Grundlage f"ur Darstellungen des \glqq Castellum\grqq von Aufgabe~\ref{Castellum}. Da die axonometrischen Angaben mit Hilfe der \glqq Schieberegler\grqq\ interaktiv ver"anderbar sind, k"onnen Sie viele verschiedene parallelperspektivische Darstellungen ausprobieren.
\end{Teilaufgaben}
\end{Aufgabe}
%
\begin{Loesung}
	\begin{figure}[H]
		\centering
		\includegraphics[width=0.8\textwidth]{Grafiken/UE2_H3.png}
		\caption{Vollst"andige Darstellung in Milit"arperspektive}
		\label{fig.H3}
	\end{figure}
\end{Loesung}
%
%%%%%%%%%%%%%%%%%%%%%%%%%%%%%%%%%%%%%%%%%%%%%%%%%%%%%%%%%%%%%%%%%%%%%%%%%%%%%%%
%\newpage

%%%%%%%%%%%%%%%%%%%%%%%%%%%%%%%%%%%%%%%%%%%%%%%%%%%%%%%%%%%
\begin{Aufgabe}[Koordinatenquader -- Koordinatenbestimmung]
Übertragen Sie die folgende Figur auf Ihr Papier, zeichnen Sie den Koordinatenquader von P und lesen
Sie die Koordinaten des Punktes P ab. Der eingezeichnete Punkt P’ soll in der Grundrissebene liegen.\\
\includegraphics[width=0.3\textwidth]{../Grafiken/Koordinatenquader-1.pdf}
%%%%%%%%%%%%%%%%%%%%%%%%%%%%%%%%%%%%%%%%%%%%%%%%%%%%%%%%%%%

Übertragen Sie die folgende Figur auf Ihr Papier, zeichnen Sie den Koordinatenquader von P und lesen
Sie die Koordinaten des Punktes P ab. Der eingezeichnete Punkt P’ soll in der Grundrissebene liegen.\\
\includegraphics[width=0.3\textwidth]{../Grafiken/Koordinatenquader-2.pdf}
\end{Aufgabe}
%
\begin{Loesung}
Keine L"osung darstellbar, dies ist eine Aufgabe um den Umgang mit GeoGebra und euer Verst"andniss f\ur Darstellungsperspektiven zu verbessern.
\end{Loesung}
%
%%%%%%%%%%%%%%%%%%%%%%%%%%%%%%%%%%%%%%%%%%%%%%%%%%%%%%%%%%%%%%%%%%%%%%%%%%%%%%%
\begin{Aufgabe}[Analyse einiger Kavalierprojektionen]
\label{Analyse_KavProj}%
%%%%%%%%%%%%%%%%%%%%%%%%%%%%%%%%%%%%%%%%%%%%%%%%%%%%%%%%%%%%%%%%%%%%%%%%%%%%%%%
\textbf{Untersuchen Sie} einige spezielle Kavalierprojektionen. Die Bildebene ist jeweils die $x_2$-$x_3$-Ebene. Die Projektionsrichtung liegt immer in derjenigen Ebene , welche die $x_1$-Achse enth"alt und den Winkel zwischen der $x_2$-Achse und der $x_3$-Achse halbiert. Diese Ebene ist durch die Gleichung \,$x_2\,=\,x_3$ gegeben. Im Folgenden bezeichne $E_1'$ stets den Schnittpunkt der durch \,$E_1$\, verlaufenden Projektionsgeraden mit der Bildebene. Bei den hier untersuchten Parallelprojektionen ergibt sich stets \,$\alpha = \ang{45}$, der Skalierungsfaktor \,$s_1$\, variiert je nachdem, wie flach oder steil die Projektionsrichtung gew"ahlt wird.
\begin{Teilaufgaben}
\item Die Projektionsrichtung sei um $\ang{45}$ zur \,$x_2$-$x_3$-Ebene geneigt. Bestimmen Sie die L"ange der Strecke \,$s_1= \overline{OE_1'}$.
\item Die Projektionsgerade habe die Steigung \,$m_2 = \frac{2}{1}$.\, Bestimmen Sie die L"ange der Strecke \,$s_1=\overline{OE_1'}$\, sowie den Neigungswinkel \,$\delta_2$\, der Projektionsgeraden zur \,$x_2$-$x_3$-Ebene.
\item Die Projektionsgerade habe die Steigung \,$m_3 = \frac{1}{2}$.\, Bestimmen Sie die L"ange der Strecke \,$s_1=\overline{OE_1'}$\, sowie den Neigungswinkel \,$\delta_3$\, der Projektionsgeraden zur \,$x_2$-$x_3$-Ebene.
\item In der Schule haben Sie zur Darstellung  r"aumlicher Objekte m"oglicherweise immer wieder eine Axonometrie verwendet, bei der \,$s_1\,=\,\tfrac{\sqrt{2}}{2}$\, und \,$\alpha = \ang{45}$\, galt. Welchen Neigungswinkel $\delta$ muss die Projektionsgerade zur $x_2$-$x_3$-Ebene haben, damit sich \, $s_1\,=\,\tfrac{\sqrt{2}}{2}$\, ergibt?
\end{Teilaufgaben}
%%%%%%%%%%%%%%%%%%%%%%%%%%%%%%%%%%%%%%%%%%%%%%%%%%%%%%%%%%%%%%%%
\end{Aufgabe}
%
\begin{Loesung}
	\begin{figure}[H]
		\centering
		\includegraphics[width=0.8\textwidth]{Grafiken/UE2_H5_1.png}
		\label{fig.H5_1}
	\end{figure}
Der Punkt P kann nun mit Hilfe des gezeichneten Quaders leicht abgelesen werden.\\
P = (5,4,2);\\
\pagebreak

\begin{figure}[H]
		\centering
		\includegraphics[width=0.8\textwidth]{Grafiken/UE2_H5_2.png}
		\label{fig.H5_2}
	\end{figure}
Der Punkt P kann nun mit Hilfe des gezeichneten Quaders leicht abgelesen werden.\\
P=(3,3,-2)
\end{Loesung}
%
%%%%%%%%%%%%%%%%%%%%%%%%%%%%%%%%%%%%%%%%%%%%%%%%%%%%%%%%%%%%%%%%
\begin{Aufgabe}[Fortsetzung der Analyse von Aufgabe~\ref{Analyse_KavProj} -- Projektion auf verschiedene Ebenen\label{KavProj_AndereBildebene}]
\begin{Teilaufgaben}
\item Die Bildebene werden nun so \glqq tiefergelegt\grqq, dass sie \textbf{parallel zur}\, $x_2$-$x_3$-Ebene ist und den Punkt\, $(-1,0,0)$\, enth"alt. Untersuchen Sie, wie die Punkte\, $O$,\, und\, $E_1$\, projiziert werden, wenn die Projektionsrichtung so ist wie in der ersten Teilaufgabe von \ref{Analyse_KavProj}. Ist der Skalierungsfaktor\, $s_1$\, im Vergleich zur Aufgabe~\ref{Analyse_KavProj} ver"andert?
\item Was ver"andert sich, wenn Sie die Bildebene\, $\pi$\, um eine weitere L"angeneinheit \glqq tieferlegen\grqq, so dass sie den Punkt\, $(-2,0,0)$\, enth"alt.
\item Versuchen Sie einen Ergebnissatz zu formulieren, der die von Ihnen beobachteten Sachverhalte bei Parallelprojektionen zusammenfasst:\\[1ex]
\textit{Wenn man bei gegebenem r"aumlichen Dreibein \, $\bigl(\,O;\; E_1,\, E_2,\, E_3\,\bigr)$\, und gegebener Projektionsrichtung unterschiedliche zueinander parallel liegende Bildebenen w"ahlt, so ver"andert sich zwar die Lage \ldots, die \ldots bleiben jedoch unver"andert.}
\end{Teilaufgaben}
%%%%%%%%%%%%%%%%%%%%%%%%%%%%%%%%%%%%%%%%%%%%%%%%%%%%%%%%%%%%%%%%
\end{Aufgabe}
%
\begin{Loesung}
\begin{figure}[H]
	\centering
	\includegraphics[width=.5\textwidth]{Grafiken/UE2_H6_1.png}
	\caption{
		Rote Ebene: Bildebene \ensuremath{(x_1 = 0)}\\
		Blaue Ebene: Menge der m"oglichen Projektionsgeraden \ensuremath{(x_2 = x_3)}\\
		Violette Gerade: Schnittgerade der beiden Ebenen; diese bezeichnen wir als die Menge aller m"oglichen Punkte \ensuremath{E_1'}.\\
		Rote Achse: \ensuremath{E_1}\\
		Blaue Achse: \ensuremath{E_3}\\
		Gr"une Achse: \ensuremath{E_2}\\
		}
		\label{fig.H6}
\end{figure} 
\begin{Teilloesungen}
Wir bestimmen die L"ange der Strecke $s_1 = \overline{OE_1'}$:
\begin{figure}[H]
	\centering
	\includegraphics[width=.3\textwidth]{Grafiken/UE2_H6_2.png}
	\caption{Gleiche Farbgebung wie Abb. 1; Dreibein nach hinten geklappt, sodass \ensuremath{x_1} jetzt nach oben zeigt.\\
	Rote Gerade: Ortsgerade des Projektionsvektors}		
	\label{fig.H6_2}
\end{figure}
\ensuremath{\delta} beschreibt den Winkel zwischen \ensuremath{\overrightarrow{p}} und der Bildebene. Sowohl der 	Winkel, als auch die Projektionsgerade innerhalb der \ensuremath{(x_2 = x_3)}-Ebene. \\		
		Legt man die Projektionsgerade nun an den Punkt \ensuremath{E_1} an, erh"alt man den Schnittpunkt \ensuremath{E_1'} mit der Bildebene. Dementsprechend muss \ensuremath{s_1 = \overline{OE_1'} = 1} sein.
\begin{figure}[H]
	\centering
	\includegraphics[width=.5\textwidth]{Grafiken/UE2_H6_3.png}
	\caption{
	Wir blicken von unten auf die \ensuremath{x_2 = x_3}-Ebene (blau dargestellt) \\
	Violette Gerade: Schnittgerade der beiden Ebenen \\
	Rote Gerade: Ortsgerade des Projektionsvektors}
	\label{fig.H6_3}
\end{figure} 

\item \begin{figure}[H]
	\centering
	\includegraphics[width=.5\textwidth]{Grafiken/UE2_H6_4.png}
	\caption{Die Steigung der Projektionsgerade innerhalb der \ensuremath{x_2 = x_3}-Ebene betr\agt \ensuremath{m = \frac{2}{1}}}
\end{figure} 

Der Schnittpunkt der Projektionsgeraden mit der Bildebene ist einfach aus der Abbildung abzulesen. Wir sehen, dass \ensuremath{s_1 = \frac{1}{2}} ist. Um den Winkel \ensuremath{\delta} zu berechnen, nutzen wir den Satz des Pythagoras.
\begin{align*}
s_1 &= \overline{OE_1'}\\
s_1 &\coloneqq \frac{1}{2}\\
\tan(\delta) &=\frac{\textbf{Gegenkathete}}{\textbf{Ankathete}}\\
\delta &= \tan^{-1}\Big(\frac{2}{1}\Big)\\
\delta &= 63.43^\circ
\end{align*}

\item \begin{figure}[H]
	\centering
	\includegraphics[width=.8\textwidth]{Grafiken/UE2_H6_5.png}
	\caption{Die Steigung der Projektionsgerade innerhalb der \ensuremath{x_2 = x_3}-Ebene betr\agt \ensuremath{m = \frac{1}{2}}}
	\label{fig.H6_4}
\end{figure} 

\begin{align*}
s_1 &= \overline{OE_1'}\\
s_1 &\coloneqq 2\\
\tan(\delta) &=\frac{\textbf{Gegenkathete}}{\textbf{Ankathete}}\\
\delta &= \tan^{-1}\Big(\frac{1}{2}\Big)\\
\delta &= 26.57^\circ
\end{align*}

\item Wir bestimmen den Neigungswinkel $\delta$:
\begin{figure}[H]
	\centering
	\includegraphics[width=.8\textwidth]{Grafiken/UE2_H6_6.png}
	\label{fig.H6_5}
	\caption{Die Steigung der Projektionsgerade innerhalb der \ensuremath{x_2 = x_3}-Ebene betr\agt 
	\ensuremath{m = \frac{\frac{\sqrt{2}}{2}}{1}}.}
\end{figure} 
\begin{align*}
s_1 &= \overline{OE_1'}
s_1 &\coloneqq \frac{\sqrt{2}}{2}
\tan(\delta) &= \frac{\textbf{Gegenkathete}}{\textbf{Ankathete}}
\tan(\delta) &= \frac{\frac{2}{\sqrt{2}}}{1}
\delta &= \tan^{-1}\Big(\frac{2}{\sqrt{2}}\Big)
\delta &= 54.74 ^\circ
\end{align*}
\end{Teilloesungen}
\end{Loesung}